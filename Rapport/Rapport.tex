\documentclass[a4paper,11pt]{article}
\usepackage{a4wide}%

\usepackage{tabularx}

\usepackage{fullpage}%
\usepackage[T1]{fontenc}%
\usepackage[utf8]{inputenc}%
\usepackage[main=francais,english]{babel}%

\usepackage{graphicx}%
\usepackage{xspace}%
\usepackage{float}

\usepackage{url} \urlstyle{sf}%
\DeclareUrlCommand\email{\urlstyle{sf}}%

\usepackage{mathpazo}%
\let\bfseriesaux=\bfseries%
\renewcommand{\bfseries}{\sffamily\bfseriesaux}

\newenvironment{keywords}%
{\description\item[Mots-clés.]}%
{\enddescription}

\usepackage[backend=biber, style=authoryear-comp]{biblatex}
\bibliography{Biblio}{}


\newenvironment{remarque}%
{\description\item[Remarque.]\sl}%
{\enddescription}

\font\manual=manfnt
\newcommand{\dbend}{{\manual\char127}}

\newenvironment{attention}%
{\description\item[\dbend]\sl}%
{\enddescription}

\usepackage{listings}%

\lstset{%
  basicstyle=\sffamily,%
  showstringspaces = false,
  columns=fullflexible,%
  language=c,%
  frame=lb,%
  frameround=fftf,%
}%

\lstMakeShortInline{|}

\parskip=0.2
\baselineskip

\sloppy

%opening
\title{Application des techniques d'appentissage profonds à l'étude de la
  classification de protéines}
\author{Rémy Sun}



\begin{document}

\maketitle

\begin{abstract}
  
  \begin{keywords}
    Deep Learning; Protéines; Séquence
  \end{keywords}
\end{abstract}

\section*{Introduction}

Les techniques dites d'apprentissage profond ont permis de grand progrès dans de
nombreux domaines qui vont de la reconnaissance d'image à l'étude de langages
naturels (\cite{DBLP:journals/corr/ChoMGBSB14}, \cite{socher2011semi}, \cite{NIPS2014_5346}). Néanmoins, les
protéines demeurent un domaine relativement inexploré par l'apprentissage
profond à l'exception de quelques travaux
(\cite{Spencer:2015:DLN:2817095.2817106}) malgré un certain nombre de travaux
sur l'expression des gènes (\cite{Gupta031906}).

L'applications de techniques dévelloppées en langage naturel ou dans d'autres
branches de la bioinformatiques permettent de faire des recoupement sémantiques,
de la classification ou même de la prédiction sur des chaînes de \og mots\fg ou
d'acides aminés. De fait, il semble raisonnable de penser qu'une étude des
protéines avec des techniques similaires devraient permettre de similaires
progrès dans la compréhension des mécanismes gouvernants le fonctionnement des
protéines dont la séquence peptidique est fortement liée au fonctionnement.

Nous nous sommes donc interessés dans ce stage à l'applications de telles techniques
aux familles de protéines. Notre intêret ne portait pas seulement sur la
classification de telles familles ou la reconstitution d'une proteine à partir
d'un séquençage bruité, mais aussi sur l'étude de la possibilité de comprendre
des mécanismes des protéines à partir des représentations intermédiaires acquises par les
algorithmes d'apprentissages profonds entraînés.

\section{Apprentissage profond?}

\paragraph{Une technique d'apprentissage machine}

Comme toutes les techniques d'apprentissage machine, l'apprentissage profond
vise à entrainer un système pour qu'il résolve des situations sans que tous les
paramètres nécessaires à la résolution du problème n'aient été calculés par
l'implémentateur. Traditionellement, la façon de procéder serait d'utiliser par
exemple un réseau de neurones pour faire un perceptron: il y a un couche entrée,
une couche cachée et une couche sortie. Entre chaque couche une transformation
paramétrée par des variables est effectuée et en sortie on évalue par une
fonction de score le résultat renvoyé. Une optimisation sur les paramètres est
ensuite effectuée, par descente de gradient par exemple.

Ce qui différencie le Deep Learning de ces techniques d'aprentissage dites \og
creuses\fg est d'utiliser plusieurs couches cachées (d'où la notion de
profondeur). Cela augmente remarquablement l'abstraction du problème et permet
de mieux traiter les problèmes dits compositionnels, c'est-à dire qui peuvent se
décomposer en plusieurs composantes.

\subsection{Entrainement non supervisé}

\paragraph{Autoencodeur}

Comment faire pour demander à une machine de s'entrainer par elle même sans
qu'on lui dise quelle conclusion tirer à partir de son résultat? Une façon de
faire consiste à lui demander de retrouver son entrée. Ainsi, on ``encode''
l'information dans un premier temps, puis on la décode: c'est l'auto-encodeur.

Traditionellement, c'est à cela (ou aux réseaux basés sur les machines
restreintes de Boltzmann) qu'on fait référence quand on parle d'entraînement
non-supervisé. A strictement parlé, il s'agit plus d'un entraînement
auto-supervisé que non-supervisé, mais aucune réelle méthode d'entrainement
non-supervisée n'existe (il faudrait probablement revoir la structure même de
l'entrainement).

\paragraph{Donc on entraîne un réseau complexe à... retrouver l'identité?}

Ce n'est pas tant le but qui est intéressant ici, mais la façon de l'atteindre.
Evidemment, il y a un réel risque que cette manière de l'atteindre ne soit pas
particuliérement intéressante non plus (on encode l'identité, puis on décode
l'identité). Pour éviter cela on dipose de plusieurs moyens:

\begin{itemize}
\item Forcer l'encodeur à encoder l'entrée en une représentation intermédiaire
  de dimension inférieure. Ainsi, on s'assure de ne pas pouvoir juste propager
  l'identité. Mieux encore, la représentation intermédiaire, ``condensée'' peut
  permettre de découvrir des ``points robustes'' de ce qu'on est en train
  d'apprendre. Ce qui caractérise une famille de protéines par exemple...
  Néanmoins, le grand défaut de cette approche est que la perte d'information
  est ici inévitable.
\item Corrompre l'information donnée en entrée (comme proposé par \cite{Vincent:2008:ECR:1390156.1390294})pour forcer l'encodeur à
  ``deviner'' ce qu'il manque, ce qui permettrait par exemple de mettre en
  lumière des corrélations non évidentes à l'oeil nu.
\item Poser des contraintes sur la représentation intermédiaire acquise
\item Aléatoirement désactiver certains neurones, ce qui force le système à
  introduire de la redondance (et donc à choisir ce qui est réellement
  important). C'est une technique aussi employée dans d'autres réseaux neuronaux
  pour éviter de spécialiser le réseau sur l'ensemble d'entraînement et non sur
  le domaine où on veut l'appliquer. Il est raisonnable qu'un auto-encodeur
  entrainé sur des familles de protéines gére mal les chaînes purement
  aléatoires, mais il serait problèmatique que cet auto-encodeur n'arrive pas à
  traiter des protéines proches des exemples d'entraînement mais n'en faisant
  pas partie. 
\end{itemize}

Une partie de notre travail s'est focalisée sur l'étude de ces représentations
intermédiaires et sur ce qu'elles nous apprennent sur les familles de protéine
ayant servi à l'entraînement de l'auto-encodeur.

\subsection{Réseaux convolutifs}

\paragraph{Il y a un motif caractéristique dans les images de chameaux}

C'est probablement intéressant pour classifier des images d'animaux, mais de là
à le repérer sur un pixel... Ce qu'on peut faire par contre, est \og
scanner\fg des carrés $3\times 3$. En pratique, c'est créer une couche cachée
dont chaque neurone applique une même transformation sur les 9 neurones d'un
carré $3\times 3$. Il s'agit donc de faire une convolution sur l'image d'une
fonction de transformation!

Nous venons de décrire une couche générant une image réduite dont chaque
neurone/pixel donne la valeur de l'application d'une fonction à un carré
$3\times 3$. Nous appellerons une telle représentation \texttt{feature map}:
elle donne la représentation d'une \og feature\fg sur l'image.

\paragraph{Pourquoi s'arrêter à une \texttt{feature map}?}

Typiquement on crée plusieurs \texttt{feature maps} en parrallèle dans un réseau
convolutionnel. Rappellons déjà que nous n'avons pas beaucoup de contrôle sur la
caractèristique que le réseau va retrouver puisqu'il va apprendre la
caractéristique qui donne le meilleur résultat par lui-même. Multiplier les
\texttt{feature maps} permet de distinguer plus de points caractéristiques qui
seront ensuite traités par une couche supérieure.

Si on veut rajouter une couche convolutionnelle au dessus, elle opérera non
seulement une opération sur un carré de chaque \texttt{feature map}, mais elle
effectura une opérations sur les résultats de cette opération sur chaque feature
map.

\paragraph{Réduire la charge de calcul}

Pour augmenter un peu la robustesse du système à une translation par exemple, on
effectue un \texttt{pooling} qui consiste à regrouper ensemble des blocs de
neurones. Non seulement cela permet de réduire la taille de la représentation
considérée, cela fait qu'une simple translation a moins de chance de changer
quoi que ce soit à l'image obtenue après transformation.

\subsection{Réseaux récurrents}

\paragraph{Un unique neurone peut constituer un réseau profond!}

L'idée est que l'entrée est traitée comme une suite d'entrées:
typiquement c'est le cas d'une phrase par exemple. Chaque mot est traité par la
couche récurrente qui va calculer deux choses à partir de cette entrée: une
sortie \og publique\fg et une sortie caché qui détermine un paramètre de la
couche (une sorte d'état interne caché). Le second élément de la suite est
ensuite traité par ce même neurone dont l'état caché a évolué suite au
traitement du premier élément de la suite.

Si on \og déplie\fg l'exécution du problème, on réalise que la séquence passe en
fait par un nombre de couches cachées égal au nombre d'éléments dans la suite!
Ainsi, nous avons établit un réseau qui est capable de traiter un élément d'une
suite en se rappelant en quelque sortes de ce qui s'est passé avant. La seule
question qui reste à traiter est le fonctionnement exact de la couche cachée.

Il est possible de faire fonctionner cette couche cachée comme un réseau neuronal
complétement relié classique avec une sortie et entrée supplémentaire
représentant l'état caché, mais cela pose d'est problème d'évanouissement du
gradient lors de la rétropropagation.

\paragraph{Long Short-Term Memory (LSTM)}

Une architecture de couche dont le comportement naturel est de se rappeler de ce
qui s'est passé avant est exposée dans . L'idée est qu'on a un état de
cellule, un état caché et une entrée. A chaque passage dans la couche, on calcul
quel degré d'information il faut retenir de l'état de la cellule en fonction de
l'état caché et de l'entrée. Ensuite on calcule s'il faut ajouter de
l'information à l'état de la cellule. Enfin, on fait le calcul de la sortie et
de l'état caché à partir des trois données précedentes.

\paragraph{Pourquoi utiliser un réseau récurrent?}

Les réseaux récurrents se sont avérés très utiles à l'études du langage naturel
puisqu'il permettent de détecter des corrélations dans une phrase qu'un réseau
convolutionnel ne détecterait pas forcément (la fenêtre de détection d'une
couche convolutionnelle est après tout de taille finie).  La sortie récupérée
peut être de deux types différents: on peut récupérer la suite des sorties du
réseau récurrent, ou on peut choisir de récupérer la dernière sortie de la
couche récurrente qui servirait alors de \og résumé\fg de ce qui s'est passé à
la lecture de la suite.

\paragraph{Un auto-encodeur récurrent?}

Une architecture d'auto-encodeur profonde que nous avons particulièrement étudié
(et qui fournit de bons résultats pour la détection de structures comme la copie
dans une chaîne) est celle proposée dans \cite{DBLP:journals/corr/ChoMGBSB14}.
L'idée est d'utiliser un premier auto-encodeur pour servir d'encodeur: on ne
récupére que sa sortie sur le dernier terme de la suite d'entrée. Cette sortie
est de dimension finie et nous servira de résumé (qui compresse éventuellement
l'information). Il suffit ensuite de passer ce résumé dans un autre
auto-encodeur récurrent en récupérant ctte fois chaque sortie (qu'on encode
ensuite en une lettre avec un réseau dense et un \texttt{softmax}).

\subsection{Initialisation et optimisation des réseaux d'apprentissage profond}

\paragraph{Entraînement non-supervisé?}

Il a été montré dans de nombreux travaux qu'il est possible d'entrainer des
auto-encodeurs les uns à la suite et d'aboutir à des situations permettant
d'éviter de trop mauvais minima locaux. C'est d'ailleurs ce qui a motivé le
second souffle du deep learning vers la fin des années 2000

Néanmoins, des travaux plus récents ont montrés qu'il suffit d'utiliser des
initialisation aléatoires bien choisies pour largement passer outre les problème
de minima locaux: nous n'avons jamais besoin que d'une instance d'entraînement
évitant les minima locaux. Nous avons utilisé dans ce stage l'initialisation
proposée dans .

\paragraph{Rétropropagation}

Nous disposons d'une fonction de score en fin de réseau. Notre but est de
minimiser cette fonction en modifiant les paramètres interne du réseau (qui
déterminent les transformations effectuées sur l'entrée.). Pour ce faire, on
utilise typiquement le gradient de la fonction de coût. Cela peut se faire au
travers de quelquechose d'aussi simple que la descente de gradient stochastique, mais cette
dernière a de nombreux défauts. Dans cette article nous utiliserons les
optimisations dites \texttt{adagrad} et \texttt{RMSProp}.

\paragraph{De la non linéarité}

En un sens, les réseaux neuronaux cherchent à observer les données \og sous un
certain angle\fg qui permet de trouver des corrélations: cela permet de faire de
la classification, de la reconstruction, voire de la prédiction. Néanmoins si on
se borne à effectuer des opérations linéaires (sommes pondérées
d'entrés, trnasformations affines) il est tout a fait possible de passer à coté
d'un regroupement intéressant des points (les fonction linéaire préservent
notemment l'alignement ce qui peut se révéler être un grand problème). De fait,
il y a toujours \og à la fin\fg d'une couche une fonction dite \og
d'activation\fg qui opére une transformation non-linéaire sur la sortie.
Traditionnellement, on applique une sigmoïde ou une tangente hyperbolique, mais
Hinton a montré qu'il est plus judicieux d'utiliser une fonction de seuillage \texttt{ReLU}.

\section{Travail réalisé}

\subsection{Choix des architectures d'apprentissage profond}


\subsection{Clustering de protéines}

Les auto-encodeurs générent un représentation intermédiaire dans un premier
temps. Nous avons voulu étudier cette représentation intermédiaire pour en
apprendre plus sur les choses auquelles l'auto-encodeur fait attention quand il
encode la protéine et qui lui permettent de reconstruire par la suite la
protéine.

En première approximation, on pourrait penser retrouver un hyperplan ou une
nappe aparentable à une feature connue. Néanmoins une telle étude est difficile
à exploiter. Nous nous sommes penchés sur la piste du clustering
de protéines en fonction de leur représentation intermédiaire. En effet, si on
regroupe ensemble des protéines \og proches\fg du point de vue de leur
représentation intermédiaire, il est peut-être possible de gagner de remarquer
des points remarquables dans la structure de protéines qui pourront être
ré-utilisés de manière indépendante de l'apprentissage profond plus tard.

Un simple algorithme de K-means permet de faire le découpage en parties de
l'ensemble d'apprentissage.

\section{Vérification}

\section{Contribution}

\section*{Conclusion}

\printbibliography

\end{document}
